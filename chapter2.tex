\chapter{راهنمای نصب  و استفاده از \LaTeX \ }
%شامل \XePersian}

\section{مقدمه}
نرم‌افزار حروف‌چینی \TeX یکی از نرم‌افزارهای معروف حروف‌چینی متون علمی است
که در سطح وسیعی جهت حروف‌چینی  مجلات و کتب استفاده می‌شود. در این متن مختصر بر آنیم
که راهنمای سریعی برای نصب و استفاده از آن بیان کنیم با این امید که کاربران با پیگیری آن
به راحتی  بتوانند آن را نصب و استفاده نمایند.

قبل از این لازم است جهت واضح شدن شکل عملکرد این نرم‌افزار، اطلاعاتی در مورد آن داشته
باشیم که در ادامه به آن پرداخته می‌شود.

نرم‌افزار حروف‌چینی \TeX\  یک نرم‌افزار مجانی است که به صورت خط فرمانی کار می‌کند، به این
معنی که متن مورد نظر در یک فایل نوشته شده و سپس این فایل از طریق دستورات خط
فرمان به نرم‌افزار حروفچین \TeX\  داده می‌شود. این نرم‌افزار فایل داده شده را خوانده و بر
مبنای آن متن حروف‌چینی شده را به صورت یک فایل (مثلا \lr{PDF}) ارائه می‌کند.

دستورات خط فرمان متعددی برای استفاده از این نرم‌افزار حروفچین وجود دارد که از مهمترین 
آنها می‌توان به \lr{latex}، \lr{pdflatex} و \lr{xelatex} اشاره کرد. معمولاً ما این بخش از
نرم‌افزار حروفچین را موتور \TeX\  می‌نامیم. این خاصیت، اولین متمایز کنندۀ این نرم‌افزار
از سایر نرم‌افزارها نظیر Office  است زیرا در Office شما نتیجه نهایی را همزمان با تایپ
 می‌بینید ولی در این نرم‌افزار باید فایل را به حروفچین بدهید تا خودش شکل خروجی را آماده
 کند. عملاً به همین دلیل نیز آن را نرم‌افزار حروفچین می‌نامند، مشابه این که شما متن خام 
 خود را به یک فرد حروفچین می‌دهید تا با شکل دهی آن در قالب صفحات، آن را برای چاپ
 آماده کند.
 
 پس متن خام باید در یک ویرایشگر تایپ شده و سپس فایل حاصل (که پسوند آن .tex است)
 به برنامۀ حروفچین
 با استفاده از خط فرمان داده شود. ویرایشگرهایی وجود دارند که امکان وارد کردن متن خام
 و به طور همزمان، امکان دادن فایل به موتور \TeX\  و نشان دادن نتیجۀ حروف‌چینی را دارند. 
 اما تمام آنها بر مبنای همان دستورات خط فرمان عمل می‌کنند و هیچکدام به تنهایی و بدون
 دسترسی به یک موتور \TeX\  نمی‌توانند خروجی تولید کنند. البته هیچ وابستگی بین
 ویرایشگر و فایل تولید شده توسط آن وجود ندارد و یک فایل توسط هر کدام می‌تواند 
 تولید یا ویرایش شود یا فایل ایجاد شده توسط  یک ویرایشگر، در دیگری تغییر یابد.
 از معروف‌ترین این ویرایشگرها می‌توان به \lr{\href{https://www.winedt.com/}{WinEdit}}، 
 \href{https://biditexmaker.software.informer.com/3.1/}{Texmaker}
و  \href{https://notepad-plus-plus.org/}{Notepad++}  اشاره کرد.

برای حروف‌چینی فایل، می‌توان از طریق خط فرمان به صورت زیر عمل کرد. در ویندوز
وارد \lr{Command Prompt } شوید و به محل قرار گرفتن فایل مربوطه (همان فایل با پیوند
.tex) بروید. بسته به کاربرد خود  و شکل خروجی مورد نظر 
یکی از دستورات زیر را بزنید تا فایل خروجی مربوطه ایجاد شود. به جای filename  نام
فایل .tex گذاشته شود. 


\begin{center}
\begin{tabular}{|c|c|}
\hline
برای خروجی .dvi با فایل ورودی انگلیسی & \lr{latex filename}\\  \hline
برای خروجی .pdf با فایل ورودی انگلیسی & \lr{pdflatex filename}\\ \hline
برای خروجی .pdf با فایل ورودی فارسی یا  انگلیسی & \lr{xelatex filename}\\ \hline
\end{tabular}
\end{center}

\textcolor{red}{\bf توجه مهم:} دقت کنید که نام فایل یا فولدرهایی که فایل در آن قرار دارد فارسی
نباشد یا بین نام آنها فاصله وجود نداشته باشد. در صورت عدم رعایت این موضوع، در برخی
مواقع اجرا با مشکل روبرو می‌شود.

فایل آماده شده خام، شامل دستوراتی است که قسمتهای مختلف متن نظیر عنوان فصل و بخش
و سایر موارد را مشخص می‌کند. اگر این دستورات درست استفاده نشده باشند، حروفچین در زمان
حروف‌چینی خطا می‌دهد که پیام خطا شامل شماره خطی است که در آن خطا اتفاق افتاده است.
لذا، در این موارد باید مشابه خطاگیری از یک برنامۀ کامپیوتری، نسبت به رفع خطا اقدام کرد.
توجه کنید که وجود خطا ممکن است متن را به صورتی به غیر از آنچه مورد نظر است حروف‌چینی
کند و اگر تعداد خطاها زیاد باشد ممکن است قسمت یا کل متن را حروف‌چینی نکند و خروجی
نداشته باشد یا خروجی حاصل ناقص باشد.
 
\section{نصب موتور اصلی \TeX\ }
توزیع‌های مختلفی برای موتور \TeX\  وجود دارد که در اینجا به نصب  توزیع
معروف و مجانی آن به نام‌ \lr{\TeX{}Live}  می‌پردازیم. تاکید می‌شود که  توزیع‌های مختلف با هم سازگار هستند، به این
معنی که فایل آماده شده روی تمام توزیع‌های موتور \TeX\  \ کار می‌کند. لذا مهم نیست کدام
توزیع را برای نصب انتخاب کنید. با نصب هر کدام از این دو توزیع، به طور اتوماتیک
بسته \XePersian \  نصب می‌شود و نیاز به هیچ کار اضافی نیست. 

البته توصیه پدیدآورندگان بسته \XePersian \ %که جهت تولید متون فارسی در \TeX\  \ 
%این بسته را ارائه کرده‌اند، 
استفاده از  \lr{\TeX{}Live}  است. 
%\pagebreak
\subsection{نصب \lr{\TeX{}Live 2020}}
%\footnote{با توجه به ارائه نسخه 2015 از TeXLive، این
%نسخه از طریق وبسایت اعلام شده در این قسمت وجود دارد. اما فعلاً این نگارش مشکلاتی با \XePersian دارد. و لذا  نسخه 2014 این نرم‌افزار را که در اف تی پی دانشگاه یزد وجود دارد دانلود و نصب نمایید. 
%مشکل
%را در \href{http://qa.parsilatex.com/7838/%D9%85%D8%B4%DA%A9%D9%84%D8%A7%D8%AA-texlive-2015-%D9%88-%D8%A7%D8%B1%D8%AA%D9%82%D8%A7-%D8%A8%D8%B3%D8%AA%D9%87-%D9%87%D8%A7-%D8%AF%D8%B1-texlive-2014?show=7838#q7838}{این لینک} ببینید.
%}
سایت‌های معروف به \lr{CTAN}، سایت‌هایی هستند که  وظیفه توزیع نسخه‌های مختلف
مجانی موتور \TeX\  را انجام می‌دهند. یکی از این وبسایت‌ها در  دانشگاه یزد 
قرار دارد که در آدرس 
\href{http://ctan.yazd.ac.ir/}{http://ctan.yazd.ac.ir/}
در دسترس است.\footnote{این وبسایت به همت آقای
مهندس فاطمی از مرکز اطلاع‌رسانی و خدمات رایانه‌ای دانشگاه یزد ایجاد شده است.
% که جا دارد از ایشان در این خصوص تشکر کرد.
}



این سایت به صورت روزانه به‌روزرسانی می‌شود. می‌توان از این سایت در هر 
لحظه آخرین نگارش‌های نرم‌افزارهای مربوطه را دانلود کرد. لازم به ذکر است که در صورت اتصال
به شبکه دانشگاه یزد، برای دسترسی به این سایت نیازی به استفاده از اکانتینگ و 
اتصال به اینترنت نیست، بلکه این سایت از طریق شبکه داخلی دانشگاه در دسترس است.\footnote{فایل‌های مورد استفاده علاوه بر آدرس‌های ذکر شده، در اف تی پی دانشگاه یزد 
به آدرس \url{http://ftp.yazd.ac.ir/} و ذیل پوشه "نرم افزارهای کاربردی ویندوز->\lr{ TeX}" قرار گرفته است.
کاربران متصل به شبکه دانشگاه یزد می‌توانند بدون  نیاز به اتصال به اکانتینگ، 
این فایل‌ها را با سرعت بالا دانلود نمایند.  دوستان می‌توانند از لینک گوگل درایو زیر نیز
 برای دانلود استفاده نمایند. 
حجم فایل حدود 4 گیگابایت است.
\url{https://drive.google.com/file/d/11dnlN7r3s3lXWWwV3JQ-a8JUjj3H2yZs/view?usp=sharing}
}


برای نصب \lr{\TeX{}Live}  مراحل زیر را انجام دهید:
\begin{enumerate}
\item وارد سایت \href{http://ctan.yazd.ac.ir/}{http://ctan.yazd.ac.ir/}
 شوید و در پایین صفحه روی \lr{TeX Live}
 کلیک کنید.
 \item روی مسیر Images کلیک کنید و از فولدر باز شده فایل با نام 
 	\lr{texlive2020-20200406.iso} را دانلود کنید. دقت کنید که 8 شماره آخر فایل ممکن 
	است مختلف باشد زیرا نشان‌دهنده تاریخ ایجاد فایل است. دقت کنید که حجم این فایل
	حدود \lr{3.7} گیگا بایت است.
 \item پس از دانلود کامل، آن را با نرم‌افزار WinRaR  باز کنید و در  پوشه‌ای به نام
 \lr{TeXLive2020} فایل را Extract کنید.
 \item وارد این پوشه شوید و برنامه 
 %install-tl
\lr{install-tl-windows}
 را اجرا کنید. ادامه روند مشابه نصب سایر نرم‌افزارها 
 است. روند نصب بسته به سرعت کامپیوتر شما ممکن است تا بیش از 2 ساعت طول بکشد. توصیه می‌شود جهت افزایش سرعت نصب و جلوگیری از تداخل ممکن، آنتی ویروس خود را قبل از شروع نصب، غیرفعال کنید.
 تصویر پنجره‌های نصب به صورت زیر است (به ترتیب از چپ به راست و از بالا به پایین). دقت کنید که باید تا ظاهر شدن آخرین پنجره که حاوی پیام
 \lr{Welcome to TeXLive!} است و فعال شدن دکمه \lr{Close} منتظر بمانید. در صورت بستن پنجره قبل از این، نصب کامل نشده است و امکان استفاده وجود ندارد.
\latin
\begin{figure}[htb]
\includegraphics[width=0.35\textwidth]{TexLive2020-installer}\hfill
\includegraphics[width=0.5\textwidth]{TexLive2020-installer-2}
\vspace*{4mm}

\includegraphics[width=0.45\textwidth]{TexLive2020-installer-3}\hfill
\includegraphics[width=0.50\textwidth]{TexLive2020-installer-4}
%\vspace*{4mm}

%\includegraphics[width=0.45\TeX\ \ twidth]{TexLive2015-5}\hfill
%\includegraphics[width=0.45\textwidth]{TexLive2015-6}
%\vspace*{4mm}
%
%\includegraphics[width=0.45\textwidth]{install-6}
\persian
\caption{پنجره‌های نصب \lr{\TeX{}Live2020} (ترتیب از چپ به راست)}
\end{figure}

\persian

 \item پس از پایان نصب، موتور \TeX\  آماده استفاده است. اگر قصد استفاده از \XePersian
 \ دارید، فقط لازم است فونت‌های مربوطه را که در بالا لینک آن آمده است را نصب کنید. عملاً نصب \lr{\TeX{}Live2020} در این مرحله پایان یافته است که شامل \XePersian \ نیز هست.
\end{enumerate}

\subsection{نصب فونت‌های  فارسی}
لازم است
که فونت‌های فارسی استفاده شده در متون فارسی روی سیستم عامل نصب شده باشد. لذا
تنها کار اضافی این است که مجموعه فونت‌های جمع آوری شده در فایل زیر روی سیستم عامل
نصب شود. توصیه می‌شود حتی اگر فونت‌ها را روی کامپیوتر خود دارید، دوباره آنها را با استفاده
از فونت‌های فایل زیر رونویسی کنید. این کار از بسیاری مشکلات بعدی جلوگیری می‌کند.
\begin{latin}

\href{http://bayanbox.ir/id/4609192605141061595}{Part 1: http://bayanbox.ir/id/4609192605141061595}

\href{http://bayanbox.ir/id/5468937351173971771}{Part 2: http://bayanbox.ir/id/5468937351173971771}

\href{http://bayanbox.ir/id/4133277893427051503}{Part 3: http://bayanbox.ir/id/4133277893427051503}
\end{latin}


\persian
\section{سایر ابزارهای مفید}
\subsection{ادیتور  Notepad++}
نرم‌افزار حروف‌چین \TeX\   دارای GUI مخصوص به خود نیست و روش کار با آن به این صورت است که فایل مربوط به متن مورد نظر ابتدا با استفاده از یک ادیتور، آماده‌سازی می‌شود و سپس، فایل با استفاده از دستورات خط فرمان، به موتور \TeX\  جهت حروف‌چینی داده می‌شود و حاصل این حروف‌چینی به صورت یک فایل پی دی اف تحویل می‌گردد. ادیتورهای متعددی برای این منظور وجود دارد و قابل استفاده است. از بین آنها، به نصب ادیتور \lr{Notepad++} می‌پردازیم.

ادیتور \lr{Notepad++ }  به دلیل قابلیت فارسی‌نویسی و همچنین از راست به چپ نویسی و امکان اجرای دستورات خط فرمان در ادیتور،
انتخاب مناسبی برای نوشتن متون  است. این نرم افزار متن باز از طریق وبسایت \href{https://notepad-plus-plus.org/}{https://notepad-plus-plus.org/} قابل دریافت است.

نصب نرم‌افزار، مشابه سایر نرم‌افزارها است و پیچیدگی خاصی ندارد. پس از نصب، برای فعال کردن قابلیت اجرای دستورات خط فرمان با استفاده از کلید \lr{F6}، باید افزونه \lr{NppExec} روی این نرم‌افزار نصب شود.

برای نصب این افزونه، پس از ورود به نرم‌افزار \lr{Notepad++}، وارد  منوی \lr{Plugins} و سپس \lr{Plugins Admin} شوید تا پنجره مطابق شکل زیر باز شود. در پنجره باز شده ابتدا در قسمت \lr{Search}، کلمه \lr{Nppexec} را وارد کنید، سپس مربع کنار این افزونه را تیک بزنید\footnote{در شکل، این افزونه به دلیل نصب بودن روی \lr{Notepad++} وجود ندارد ولی در لیست شما باید باشد.} و سپس روی \lr{Install} کلیک کنید.
\latin
\begin{figure}[htb]
\includegraphics[width=0.4\textwidth]{Notepad-1}\hfill
\includegraphics[width=0.55\textwidth]{Notepad-2}
%\vspace*{4mm}
%
%\includegraphics[width=0.45\textwidth]{TexLive2020-installer-3}\hfill
%\includegraphics[width=0.50\textwidth]{TexLive2020-installer-4}
\persian
\caption{پنجره‌های نصب \lr{NppExec} (ترتیب از چپ به راست)}
\end{figure}

\persian
 حال با زدن کلید \lr{F6}  در ادیتور، پنجره اجرای دستور به صورت شکل~\ref{NPPEXEC} باز می‌شود.
 \latin
\begin{figure}[htb]
\centerline{
\includegraphics[width=0.45\textwidth]{NppExec-1}\hfill
\includegraphics[width=0.45\textwidth]{NppExec-2}
}
\persian
\caption{پنجره‌ ورود دستورات در \lr{NppExec} (ترتیب از چپ به راست)}\label{NPPEXEC}
\end{figure}
\persian 

نمونه دستوری که می‌توانید وارد کنید به صورت زیر است:

\begin{latin}
\begin{verbatim}
NPP_SAVE
cd $(CURRENT_DIRECTORY)
xelatex $(NAME_PART)
\end{verbatim}
\end{latin}
 
 می‌توان دسته دستورات مختلفی را داشت و با کلیک روی \lr{Save}، با نام‌های مختلف آن را ذخیره کرد و برای استفاده‌های بعدی داشت. 
 
 چند نکته که مفید فایده است، به شرح زیر است.
 
برای تایپ از راست به چپ کلیدهای Alt+CTRL+R  را بزنید و برای از چپ به راست نویسی کلیدهای Alt+Ctrl+L  را بزنید.

برای نیم فاصله، کلید استاندارد \lr{Ctrl+SHift+2}  است که در این ادیتور به دلیل استفاده از این ترکیب برای کار دیگری عمل نمی‌کند.
برای عمل کردن آن باید این ترکیب کلید را از ادیتور حذف کنید. برای این منظور از منوی \lr{settings -> Shortcut Mapper}
 در برگه \lr{Main Menu}  در ردیف حدودا 80  این ترکیب را پیدا کرده و به چیز دیگری (مثلا \lr{CTRL+Shift+T}) عوض کنید.
 پس از این کار، ترکیب  \lr{Ctrl+SHift+2} برای نیم فاصله (وقتی زبان صفحه کلید فارسی باشد) کار می‌کند.
 
 توجه: برای تهیه فایل مقاله یا کتاب با \XePersian، باید از کد رمزگذاری \lr{UTF8}  برای کدگذاری فایل استفاده شود. برای انتخاب در ادیتور، از منوی
 Encoding  گزینه مورد نظر انتخاب شود.
 \subsection{نرم‌افزار \lr{Sumatra PDF}}
 نرم‌افزارهای متعددی برای مشاهده فایل‌های پی دی اف وجود دارد که مشهورترین آنها \lr{Adobe Acrobat} است. در استفاده از لاتک، فایل پی دی اف مکرر تغییر می‌کند و مجدد توسط نرم‌افزار حروف‌چین بازنویسی می‌شود ولی به دلیل قفل شدن فایل پی دی اف توسط بسیاری از نرم‌افزارهای مشاهده فایل پی دی اف؛ در روند حروف‌چینی خلل ایجاد می‌شود. لذا، توصیه می‌شود از نرم‌افزار \lr{Sumatra PDF} برای مشاهده فایل‌های پی دی اف در کنار تک استفاده شود. این نرم‌افزار، متن باز است و از طریق وبسایت \href{https://www.sumatrapdfreader.org/}{https://www.sumatrapdfreader.org/} قابل دریافت است.
 %
 \subsection{قالب پیشنهادیۀ  دانشگاه یزد}
 در جهت سهولت آماده‌سازی پیشنهادیۀ کارشناسی ارشد و دکتری  دانشگاه یزد، قالب مربوطه برای زیپرشن آماده‌سازی شده است و از طریق  
 %\href{http://farshi.blog.ir/1399/09/22/\%D9\%82\%D8\%A7\%D9\%84\%D8\%A8-\%D9\%BE\%D8\%B1\%D9\%88\%D9\%BE\%D9\%88\%D8\%B2\%D8\%A7\%D9\%84-\%DA\%A9\%D8\%A7\%D8\%B1\%D8\%B4\%D9\%86\%D8\%A7\%D8\%B3\%DB\%8C-\%D8\%A7\%D8\%B1\%D8\%B4\%D8\%AF-\%D8\%AF\%D8\%A7\%D9\%86\%D8\%B4\%DA\%AF\%D8\%A7\%D9\%87-\%DB\%8C\%D8\%B2\%D8\%AF-\%D8\%AF\%D8\%B1-\%D8\%B2\%DB\%8C\%D9\%BE\%D8\%B1\%D8\%B4\%D9\%86}{این لینک }
 \href{https://yazd.ac.ir/4063-70-5324}{لینک پیشنهادیه کارشناسی ارشد}
 و
 \href{https://yazd.ac.ir/4063-70-5442}{لینک پیشنهادیه دکتری}
 در دسترس است. فیلم راهنمای استفاده نیز در دوره آموزش مقدماتی لاتک در در بخش پایانی این فصل آمده است.
 لازم به ذکر است که با توجه به تغییرات در قالب پیشنهادیۀ دانشگاه، ممکن است برخی تفاوت‌ها در فیلم و قالب فعلی باشد.
 %
 \subsection{کلاس پایان‌نامه دانشگاه یزد}
 با توجه به تنظیمات خاص پایان‌نامه‌های دانشگاه‌های مختلف و وقت قابل ملاحظه‌ای که انجام این تنظیمات از دانشجویان می‌گیرد، 
 کلاسی بر مبنای تنظیمات پایان‌نامه‌های دانشگاه یزد ایجاد شده است. این قالب از مردادماه 1398 در \lr{CTAN} قرار 
 گرفته است و به همراه نصبٍ \lr{\TeX{}Live}
 نصب می‌شود و نیاز به هیچ کاری جهت نصب نیست. 
 نمونه فایل پایان‌نامه از طریق وبسایت زیر در دسترس دانشجویان عزیز می‌باشد.
 
 \centerline{\href{http://yazd-thesis.blog.ir/}{http://yazd-thesis.blog.ir/}}
 
 در این کلاس کلیه تنظیمات مربوط به پایان‌نامه لحاظ شده و نیاز به انجام تنظیمات اضافی ندارد. 
 توضیحات استفاده از این کلاس و فایل نمونه در وبسایت فوق آمده است.
 
 لینک فیلم راهنمای استفاده نیز در دوره آموزش مقدماتی لاتک در در بخش پایانی این فصل آمده است.
 %\pagebreak
 \subsection{تبدیل فایل‌های Word  به \LaTeX\   و برعکس} \label{sec:grineq}
 یک نرم‌افزار قوی برای تبدیل بین Word  و \LaTeX، نرم‌افزار \href{http://www.grindeq.com/}{GrindEQ}  است.\footnote{این راهنما بر مبنای ورژن این نرم‌افزار در سال 1395 تهیه شده است. استفاده ورژن‌های جدید ممکن است کمی متفاوت باشد.}
 این نرم‌افزار مجانی نیست ولی تا 10 فایل را برای شما تبدیل می‌کند. برای انجام تبدیل لازم است نرم‌افزار \lr{Office}  را نصب و سپس
 دو فایل مربوط به تبدیل بین Word  و \LaTeX  (در پوشه Convert وجود دارد) را نصب نمایید.
 برای نسخه‌های بعدی \lr{Office}، آخرین نگارش نرم‌افزار را از وبسایت فوق دریافت و نصب نمایید. نرم‌افزار منطبق بر آفیس را (از حیث 32 بیتی یا 64 بیتی بودن) را نصب کنید وگرنه عملکرد تبدیل مناسب نخواهد بود.
 \subsubsection{تبدیل Word به \LaTeX}
 فایل خود را در Word باز کنید. سپس از منوی \lr{File}،
 \lr{Save As}  را انتخاب کنید و در قسمت \lr{Save as Type}، نوع \lr{LaTeX[GrindEQ]}
 را انتخاب کنید.  از پنجره باز شده گزینه‌های مناسب را انتخاب و فایل را ذخیره کنید. فایل ذخیره شده با فرمت \LaTeX \  است.
 
 اگر فایل Word شما \textcolor{red}{\bf فارسی} است، باید از پنجره باز شده، در قسمت encoding، 
 گزینۀ \lr{UTF8} یا \lr{Unicode}  را انتخاب کنید. (شکل~\ref{Grindeq} را ببینید.) در فایل
 \LaTeX \ ایجاد شده نیز باید دستور \verb+ \usepackage{xepersian}+  و دستورات مربوط به فونت متن اضافه شود.
 
 همچنین در قسمتهای انگلیسی باید دستور \verb+\latin+  قبل از متن انگلیسی و \verb+\persian+ بعد از متن انگلیسی قرار گیرد.
 
 \subsubsection{تبدیل \LaTeX \ به Word}
  در Word   فایل \LaTeX  را باز کنید (فایل با پسوند \lr{.tex}). از پنجره باز شده مشابه قبلی، گزینه‌های مناسب را انتخاب کنید. فایل  به فرمت 
  Word تبدیل شده و باز می‌شود و می‌توانید آن را ذخیره کنید. امکان انتخاب فونت و سایر خواص در پنجره باز شده هنگام تبدیل ممکن است.
  
  مشابه قبل، اگر فایل \LaTeX \ شما \textcolor{red}{\bf فارسی} است، باید از پنجره باز شده، در قسمت encoding، 
 گزینۀ \lr{UTF8} یا \lr{Unicode}  را انتخاب کنید. (شکل~\ref{Grindeq} را ببینید.)  


\latin
\begin{figure}
\begin{center}
\includegraphics[width=0.45\textwidth]{Word2Latex} %\hfill
\includegraphics[width=0.45\textwidth]{Latex2Word}
\end{center}
\persian
\caption{پنجره تبدیل Word  به \LaTeX\  (سمت  چپ ) و \LaTeX\  به Word (سمت راست).}\label{Grindeq}
\end{figure}

  \persian 

%{\bf توجه:} با توجه به این که امکان تبدیل فایل‌های   \lr{Farsi\TeX\ } به یونیکد در زیر بیان شده است، می‌توان پس از تبدیل
%این فایل‌ها به یونیکد، آنها را با استفاده از ابزار فوق به Word  نیز تبدیل کرد. البته این مورد از نظر کیفیت انجام آزمایش نشده است.
 \subsection{جزئیات فارسی نویسی در \lr{IPE Drawing}}
نرم‌افزار آزاد (مجانی) \href{http://ipe.otfried.org/}{\lr{Ipe Drawing}} که یک ابزار قوی برای رسم اشکال است و بر مبنای \TeX\ کار می‌کند 
نسخه جدید خود را منتشر کرد. این نسخه شامل فایل باینری برای سیستم‌های ویندوزی نیز هست. این نگارش 
جدید را 
%علاوه بر 
از وبسایت آن به آدرس \url{http://ipe.otfried.org/}
  می‌توانید 
  %از لینک زیر نیز 
  دریافت کنید. این ابزار امکان درج فرمول‌های ریاضی و متون فارسی و همچنین درج تصاویر 
  با فرمت‌های bmp  و jpg  را نیز می‌دهد. لازم به ذکر است که برای اجرای این نرم‌افزار، حتما باید یکی از نگارش‌های TeX
  (نظیر \lr{Mik\TeX\ }  یا \lr{\TeX{}Live}) روی کامپیوتر شما نصب باشد. تصاویر تولید شده توسط این
  نرم‌افزار کاملاً برداری \lr{(vector)} است و حجم آن نیز پایین است.

  برای نصب نرم‌افزار کافی است فایل نصبی را از وبسایت دانلود و آن را در مسیری باز کنید. سپس در
  مسیر bin از مسیر ایجاد شده فایل ipe را اجرا کنید. 
  آخرین نگارش این نرم‌افزار در حال حاضر \lr{Ipe 7.2.24} است و نسخه 64 بیتی آن از طریق لینک 
  \href{https://github.com/otfried/ipe/releases/download/v7.2.24/ipe-7.2.24-win64.zip}{\lr{ipe-7.2.24-win64.zip}}
  و نسخه 32 بیتی آن از طریق لینک \href{https://github.com/otfried/ipe/releases/download/v7.2.24/ipe-7.2.24-win32.zip}{\lr{ipe-7.2.24-win32.zip}} 
  قابل دریافت است.
%\href{http://bayanbox.ir/download/5027279069265890785/ipe-7.2.2-win.zip}{  دریافت \lr{Ipe Drawing 7.2.2-Win}}
\subsubsection{فارسی نویسی با استفاده از بسته زیپرشن}
برای استفاده از این امکان لازم است ورژن Ipe مورد استفاده حداقل 7.2.2 باشد. 
%برای گرفتن آن می‌توانید از این \href{http://bayanbox.ir/download/5027279069265890785/ipe-7.2.2-win.zip}{لینک} دانلود نمایید یا به سایت اصلی Ipe به آدرس  \url{http://ipe.otfried.org/} مراجعه نمایید. آخرین نگارش این نرم‌افزار در حال حاضر \lr{Ipe 7.2.23} است.
%
مزیت اصلی این روش نسبت به روش قبلی، ساده تر بودن استفاده از آن و همچنین امکان استفاده از انواع مختلف فونت فارسی است.

الف) دستورات زیر را در قسمت \lr{Latex Preamble} در منوی \lr{Edit} در \lr{Document Properties} قرار دهید:

\latin
\begin{verbatim}
\usepackage[RTLdocument=off]{xepersian}
\settextfont{Yas}
\setdigitfont{Yas}
\end{verbatim}
\persian

به جای فونت \lr{Yas}  می‌توانید از هر فونت دیگری که روی کامپیوتر شما نصب است استفاده کنید. بهتر است فونت، با فونت متنی که میخواهید شکل را استفاده کنید یکسان باشد.

ب) در همین پنجره و خط بالای \lr{LaTeX preamble}، گزینه \lr{xetex} را برای \lr{Latex Engin} استفاده کنید.

ج) برای درج متن؛ هیچ دستور خاصی نیاز نیست. اما اگر چند کلمه به عنوان متن دارید باید آن را داخل دستور \verb+\rl{}+ قرار دهید وگرنه ترتیب کلمات برعکس می‌شود.

نمونه فایل ایجاد شده از لینک زیر در دسترس است:

\centerline{\href{http://bayanbox.ir/download/7805431603599153524/Ipe-7.2.2-Xepersian-Farsi-Sample.pdf}
{\lr{Ipe-7.2.2-Xepersian-Farsi-Sample.pdf}}}

این نمونه، از نگارش \lr{Ipe-7.2.8} کار نمی‌کند و خطا می‌گیرد. برای رفع این مشکل، کافی است در قسمت \lr{Latex Preamble}، دستور زیر را در ابتدای دستورات قبلی قرار دهید:
\latin
\begin{verbatim}
\ipedefinecolors{}
\end{verbatim}
\persian
برای سادگی، دستورات \lr{Latex Preamble} باید به صورت زیر باشد:

 
\latin
\begin{verbatim}
\ipedefinecolors{}
\usepackage[RTLdocument=off]{xepersian}
\settextfont{Yas}
\setdigitfont{Yas}
\end{verbatim}
\persian
 

نمونه قبلی به صورت زیر بروزرسانی شد و از نگارش 7.2.8 به بعد، قابل استفاده است.

\centerline{\href{http://bayanbox.ir/download/5817211611299891045/Ipe-7.2.8-Xepersian-Farsi-Sample.pdf}
{\lr{Ipe-7.2.8-Xepersian-Farsi-Sample.pdf}}}

روش دیگر برای حل مشکل، اضافه کردن \lr{style sheet} با نام \lr{right-to-left.isy}  است. صرفاً با اضافه کردن این \lr{style sheet} به فایل قبلی، فایل بدون مشکل ایجاد خواهد شد و نیاز به دستور اضافه شده فوق ندارد. لازم به ذکر است که این \lr{style sheet} از ورژن \lr{Ipe-7.2.16} در فولدر \lr{styles} از محل نصب نرم‌افزار قرار دارد و در نسخه های قبلی نیامده است. مزیت دیگر استفاده از \lr{style sheet} این است که برای متون گزینه \lr{rtl} نیز در پنجره درج متن اضافه می‌شود که با انتخاب آن، نیازی به استفاده از دستور \verb+\rl{}+ برای متون فارسی نیست. این روش به نظر اصولی‌تر است. نمونه فایل به صورت زیر است:

\centerline{\href{http://bayanbox.ir/download/7989814166958212055/Ipe-7.2.8-Xepersian-Farsi-Sample-Style-sheet.pdf}
{\lr{Ipe-7.2.8-Xepersian-Farsi-Sample-Style-sheet.pdf}}}


%\begin{figure}[htb]
%\begin{figure}
%\begin{center}
%\includegraphics[width=0.3\textwidth]{Ipe-7.2.2-Xepersian-Farsi-Sample} \hfill
%\includegraphics[width=0.3\textwidth]{Ipe-7.2.8-Xepersian-Farsi-Sample} \hfill
%\includegraphics[width=0.3\textwidth]{Ipe-7.2.8-Xepersian-Farsi-Sample-Style-sheet} %\hfill
%\includegraphics[width=0.45\textwidth]{Latex2Word}
%\end{center}
%
%\caption{نمونه شکل فارسی تولید شده با استفاده از بستۀ \lr{arabi}. دریافت فایل از \href{http://bayanbox.ir/id/8287021624570840892?info}{دریافت فایل نمونه PDF}}
%\end{figure}




\section{منابع آموزشی و فایل‌های نمونه}
%برای فایل‌های آموزشی و فایل نمونه، به لینک زیر مراجعه کنید:

\begin{tabular}{|p{0.25\textwidth}|p{0.75\textwidth}|}\hline
فیلم‌های آموزشی دوره مقدماتی لاتک و زیپرشن & \footnotesize %\url{http://farshi.blog.ir}
\href{http://farshi.blog.ir/1399/08/04/LaTeX-XePresian-Course}{\lr{http://farshi.blog.ir/1399/08/04/LaTeX-XePresian-Course}}
\\ \hline 
%قالب پایان‌نامه دانشگاه یزد & \footnotesize %\url{http://farshi.blog.ir}
%\href{https://yazd-thesis.blog.ir/page/%D8%AF%D8%B1%DB%8C%D8%A7%D9%81%D8%AA-%D9%82%D8%A7%D9%84%D8%A8-%D9%BE%D8%A7%DB%8C%D8%A7%D9%86-%D9%86%D8%A7%D9%85%D9%87-yazd-thesis}{\lr{https://yazd-thesis.blog.ir/page/%D8%AF%D8%B1%DB%8C%D8%A7%D9%81%D8%AA-%D9%82%D8%A7%D9%84%D8%A8-%D9%BE%D8%A7%DB%8C%D8%A7%D9%86-%D9%86%D8%A7%D9%85%D9%87-yazd-thesis}}
%\\ \hline 
لینک راهنمای نصب، قالب پایان‌نامه و پروپوزال دانشگاه یزد & \footnotesize %\url{http://farshi.blog.ir}
\href{https://zil.ink/tex}{\lr{https://zil.ink/tex}}
\\ \hline 
فایل نمونه اسلاید انگلیسی با بسته بیمر & \footnotesize %\url{http://farshi.blog.ir}
\href{https://bayanbox.ir/download/5666582903691050488/Sample-Slide-English.zip}{\lr{\small https://bayanbox.ir/download/5666582903691050488/Sample-Slide-English.zip}}
\\ \hline 
فایل نمونه اسلاید فارسی با بسته‌های  زیپرشن و  بیمر & \footnotesize %\url{http://farshi.blog.ir}
\href{https://bayanbox.ir/download/6309877269920220013/Slide-Farsi-Sample.zip}{\lr{\small https://bayanbox.ir/download/6309877269920220013/Slide-Farsi-Sample.zip}}
\\ \hline 
فیلم‌های آموزشی لاتک و زیپرشن (دکتر مس‌فروش)& \footnotesize %\url{http://farshi.blog.ir}
\href{http://mesforush.staff.shahroodut.ac.ir/category/\%D8\%A2\%D9\%85\%D9\%88\%D8\%B2\%D8\%B4-latex/}{\tiny \lr{http://mesforush.staff.shahroodut.ac.ir/category/\%D8\%A2\%D9\%85\%D9\%88\%D8\%B2\%D8\%B4-latex/}}
\\ \hline 
سایت پارسی-لاتک برای طرح سوالات و مشکلات & \footnotesize %\url{http://farshi.blog.ir}
\href{http://qa.parsilatex.com/}{ \lr{http://qa.parsilatex.com/}}
\\ \hline 
\end{tabular}

\begin{pspicture}(0.8in,0.8in)
\psbarcode{http://pws.yazd.ac.ir/farshi/TeX-Inst-Guide.pdf}{}{qrcode}
 \end{pspicture}
 \footnote{آخرین نگارش این راهنما از طریق لینک 
\href{http://pws.yazd.ac.ir/farshi/TeX-Inst-Guide.pdf}{\lr{http://pws.yazd.ac.ir/farshi/TeX-Inst-Guide.pdf}}\ 
قابل دانلود است.}
\hfill   
 \begin{pspicture}(0.8in,0.8in)
\psbarcode{https://aparat.com/v/w3oj5}{}{qrcode}
 \end{pspicture}
 \footnote{فیلم آموزش نصب از طریق لینک \href{https://aparat.com/v/w3oj5}{\lr{https://aparat.com/v/w3oj5}}\ در دسترس است.}
\hfill
\begin{pspicture}(0.8in,0.8in)
\psbarcode{https://zil.ink/tex}{}{qrcode}
 \end{pspicture}
\footnote{کلیه لینک‌ها از طریق وبسایت \href{https://zil.ink/tex}{https://zil.ink/tex} نیز در دسترس قرار دارد.}


